\documentclass[Journal,letterpaper]{ascelike-new}
\WarningFilter{caption}{Unknown document class}
%% Please choose the appropriate document class option:
% "Journal" produces double-spaced manuscripts for ASCE journals.
% "NewProceedings" produces single-spaced manuscripts for ASCE conference proceedings.
% "Proceedings" produces older-style single-spaced manuscripts for ASCE conference proceedings.
%
%% For more details and options, please see the notes in the ascelike-new.cls file.

% Some useful packages...
\usepackage[utf8]{inputenc}
\usepackage[T1]{fontenc}
\usepackage{lmodern}
\usepackage{graphicx}
\usepackage[style=base,figurename=Fig.,labelfont=bf,labelsep=period]{caption}
\usepackage{subcaption}
\usepackage{amsmath}
%\usepackage{amsfonts}
%\usepackage{amssymb}
%\usepackage{amsbsy}
\usepackage{newtxtext,newtxmath}
\usepackage[colorlinks=true,citecolor=red,linkcolor=black]{hyperref}
%
% Please add the first author's last name here for the footer:
\NameTag{Ionel Asanache, \today}
% Note that this is not displayed if the NoPageNumbers option is used
% in the documentclass declaration.
%
\begin{document}

% You will need to make the title all-caps
\title{Universal Solidarization Criterion: A New Framework for Modular Scour Protection}

\author[1]{Ionel Asanache}
\author[2]{Alexandru Cristian Ionescu}
\author[3]{Romeo Ciortan}

\affil[1]{Doctoral Researcher, Doctoral School of Applied Sciences, Ovidius University of Constanța, Romania, with corresponding author email. Email: ion.asanache@gmail.com}
\affil[2]{Doctoral Researcher, Doctoral School of Applied Sciences, Ovidius University of Constanța, Romania}
\affil[3]{PhD Supervisor, Ovidius University of Constanța, Romania}

\maketitle

% Please include an abstract:
\begin{abstract}
Local scour induced by ship propellers represents a persistent threat to the stability of quay walls and other critical marine infrastructures. Traditional design approaches often treat protective elements as isolated units, leading to oversized and resource-intensive systems. This paper introduces the \textit{Universal Solidarization Criterion} (USC), a novel analytical framework that quantifies the collective resistance of interconnected protective elements subjected to propeller jet forces. The proposed criterion introduces a solidarization coefficient ($\etaup$), which mathematically captures the reinforcement effect generated by interlinked modular units arranged in a grid structure.

The USC model builds upon and extends classical sediment transport formulations, including the Shields criterion, by incorporating inter-unit resistance and shared load distribution. Theoretical derivations are presented alongside comparative analyses with conventional design methodologies, highlighting significant reductions in required unit weight, up to 50\%, without compromising stability. While the concept is illustrated through scour protection systems, its mathematical form suggests broader applicability across structural, geotechnical, and hydraulic engineering.

The results open pathways toward more sustainable, scalable, and cost-effective erosion protection strategies. Future work will focus on CFD simulations and physical model testing to validate the USC under real-world flow conditions and geometries.\\
\textbf{Keywords}: modular scour protection; vertical quay walls; sediment transport; Shields parameter; Universal Solidarization Criterion (USC); structural interconnectivity
\end{abstract}


\section{Introduction}

Scour around hydraulic structures, such as vertical quay walls, bridge piers, and breakwaters, remains one of the most critical challenges in port and coastal engineering. Localized erosion caused by high-velocity flows, especially those generated by ship propellers during berthing and unberthing operations, can compromise structural integrity, endanger navigation safety, and require costly remediation.



Historically, the understanding of sediment transport and erosion has evolved through increasingly sophisticated models. Early empirical observations linked particle dislodgement to flow velocity, laying the conceptual groundwork for the development of more rigorous criteria. A major milestone was reached with \cite{shields1936}, who introduced a dimensionless framework for determining the onset of sediment motion based on bed shear stress, particle size, and fluid density. The Shields parameter has since become a cornerstone in sediment transport analysis, widely used in both academic and engineering applications.

Despite these advancements, classical models typically treat sediment grains or protective elements as independent entities, whose resistance is determined solely by individual weight and local hydrodynamic forces. However, modern scour protection systems---particularly modular and engineered solutions---exhibit complex behavior resulting from the physical interconnectivity and mutual reinforcement between units.

In response to this gap, the \textbf{Universal Solidarization Criterion (USC)} (Patent No. PCT/RO2025 /050001, 2025) is proposed as a new analytical framework that accounts for collective resistance. By introducing a dimensionless solidarity coefficient ($\etaup$), USC quantifies the beneficial effect of structural interdependence within modular protection systems. This approach bridges traditional hydraulic criteria with structural mechanics and opens new avenues for optimizing scour protection design in sustainable and scalable ways.


\section{Background and State of the Art}

The erosion of the seabed near vertical quay walls, particularly under the high-energy influence of propeller jets, remains an unresolved vulnerability in coastal and port infrastructure. Over time, numerous models have been developed to estimate sediment entrainment, beginning with simple velocity-based thresholds and advancing toward complex shear-stress formulations and probabilistic transport functions.


\subsection{Early Velocity-Based Thresholds}

Among the earliest methods is the \textbf{critical velocity formula}, derived from balancing hydrodynamic lift with submerged particle weight:
\[v_{crit}=\sqrt{g\cdot d_{50}\cdot \mathrm{\Delta }}\]
where:
\begin{itemize}
\item  $v_{crit}\ $is the critical velocity [m/s],

\item  $g$ is the gravitational acceleration [9.81 m/s\textsuperscript{2}],

\item  $d_{50}$ is the median grain diameter [m],

\item  $\Delta =\frac{{\rho }_s-\rho }{\rho }$ is the submerged specific gravity.
\end{itemize}

While intuitive, this model does not account for boundary layer mechanics or sediment size distributions.


\subsection{Shields' Dimensionless Criterion \protect\citeyear{shields1936}}

A major leap came with \textbf{Shields' parameter}, which introduced a non-dimensional shear stress for evaluating incipient motion:
\[\theta =\frac{\tau }{({\rho }_s-\rho )\cdot g\cdot d}\]
with:
\begin{itemize}
\item  $\tau =$ bed shear stress [Pa],

\item  ${\rho }_s$, $\rho =$ densities of sediment and fluid, respectively [kg/m\textsuperscript{3}],

\item  $d=$ characteristic sediment diameter [m].
\end{itemize}

The \textbf{critical value }${\boldsymbol{\theta }}_{\boldsymbol{cr}}$ marks the onset of sediment motion and is typically obtained from the empirical \textbf{Shields curve}, which relates ${\boldsymbol{\theta }}_{\boldsymbol{cr}}$ to the particle Reynolds number ${Re}_p$.


\subsection{Meyer-Peter \& M\"{u}ller Formula \protect\citeyear{meyer-peter1948}}

Widely used in engineering practice, this empirical formula estimates bedload transport rate per unit width:
\[q_b=8\cdot {\left(\theta -{\theta }_{cr}\right)}^{1,5}\cdot \sqrt{\left(s-1\right)\cot g\cdot d^3}\]
where:
\begin{enumerate}
\item  $q_b$ = bedload transport rate [m\textsuperscript{2}/s],

\item  $s=\frac{{\rho }_s}{\rho }$,

\item  $d=$ characteristic sediment diameter [m].
\end{enumerate}


\subsection{Einstein's Stochastic Transport Theory \protect\citeyear{einstein1950}}

Einstein proposed a probabilistic model based on individual grain entrainment:
\[q_s=\phi \cdot d\cdot \sqrt{\left(s-1\right)\cdot g\cdot d}\]
where $\phi $ is a dimensionless transport intensity coefficient depending on flow and sediment parameters.


\subsection{Engelund \& Hansen Total Load Equation \protect\citeyear{engelund1967}}

Used for finer sediments and cohesive flows:
\[q_t=0.05\cdot \frac{u^5}{g\cdot S}\]
with:

\begin{itemize}
\item  $q_t$ = total sediment transport rate [m\textsuperscript{2}/s],

\item  u = mean flow velocity [m/s],

\item  S = energy slope (bed slope).
\end{itemize}


\subsection{Limitations of Classical Models}

While valuable, all these models share key limitations when applied to \textbf{scour protection systems}:

\begin{itemize}
\item  They estimate \textbf{sediment mobility}, not the \textbf{resistance of engineered layers}.

\item  They assume \textbf{natural particle arrangements}, ignoring artificial geometries.

\item  They rarely account for \textbf{inter-element cohesion} or \textbf{solidarity} between units.
\end{itemize}


\subsection{Toward a New Paradigm: The Universal Solidarization Criterion (USC)}

Modern scour protection strategies increasingly rely on \textbf{modular artificial elements}, often deployed over geotextile substrates or interconnected via synthetic ropes. In such systems, \textbf{collective behavior}, not just individual mass, defines resistance to dislodgement.

This article introduces the \textbf{Universal Solidarization Criterion (USC or Asanache Criterion)}, which models the \textbf{effective resistance} of a modular unit as a sum of:

\begin{itemize}
\item  its own weight,

\item  cohesive force contributions from adjacent units,

\item  and the internal cohesion of the assembly.

\end{itemize}

\[G_{eff}=G_i+\eta \cdot \sum{G_n}\]
where:

\begin{itemize}
\item  $G_{eff}$ is the total resistance,

\item  $G_i$ is the unit's self-weight,

\item  $\sum{G_n}$ are neighboring weights connected via a geotextile-mesh structure,

\item  $\eta $ is the \textbf{solidarization coefficient} (0 $\mathrm{<}$ $\eta $ $\mathrm{\le}$ 1).
\end{itemize}

The USC redefines modular resistance in a way compatible with existing sediment transport theory, yet fully responsive to \textbf{artificial configurations}, \textbf{custom assemblies}, and \textbf{designable cohesion}, marking a conceptual shift in coastal defense design.


\section{Proposed Methodology -- Universal Solidarization Criterion (USC)}

Traditional scour resistance models treat sediment or protective units as isolated bodies subject to hydrodynamic forces, with dislodgement criteria based on critical shear stress or threshold velocities. However, in engineered systems---such as block mats, synthetic armor layers, or modular scour protection schemes---the assumption of independence is inadequate. The \textbf{Universal Solidarization Criterion (USC)} introduces a new approach to quantify resistance through collective interaction.

\subsection{Conceptual Foundation}

The USC is grounded in the observation that protective elements interconnected through geotextile meshes or rope-based frameworks do not behave independently. Instead, when one unit is subjected to a propeller-induced jet or turbulent flow, the adjacent units contribute to its resistance by distributing force and adding effective mass. This \textbf{solidarization effect} enhances global stability and reduces local failure probability.

\subsection{Governing Equation}

The total effective resistance $G_{eff}$ of a modular unit can be expressed as:
\[G_{eff}=G_i+\eta \cdot \sum{G_n}\]
where:

\begin{itemize}
\item  $G_{eff}$ is the total resistance,

\item  $G_i$ is the unit's self-weight,

\item  $\sum{G_n}$ are neighboring weights connected via a geotextile-mesh structure,

\item  $\eta $ is the \textbf{solidarization coefficient} (0 $\mathrm{<}$ $\eta $ $\mathrm{\le}$ 1), quantifying the degree of load-sharing via mechanical or structural linkage.
\end{itemize}

The coefficient $\eta $ a is dependent on the strength, flexibility, and pre-tension of the linking system (e.g., geotextile-embedded rope), as well as on installation conditions (e.g., soil embedment, interlocking friction).

\subsection{Force Balance with Hydrodynamic Load}

To prevent dislodgement under a hydrodynamic force $F_d$, the following inequality must hold:
\[{F_d<G}_{eff}\]
The drag force $F_d$ exerted by the propeller jet on the unit is estimated using the standard form:
\[F_d=\frac{1}{2}\cdot C_d\cdot \rho \cdot v^2\cdot A\]
Where:
\begin{itemize}
\item  $\rho $ id the density of water [kg/m\textsuperscript{3}],

\item  $C_d$ is the drag coefficient (dimensionless),

\item  $A$ is the projected area of the unit normal to the flow [m\textsuperscript{2}],

\item  $v$ is the local velocity at the unit location [m/s].
\end{itemize}

Substituting the USC relation, the full condition for stability becomes:
\[G_i+\eta \cdot \sum{G_n}>\frac{1}{2}\cdot C_d\cdot \rho \cdot v^2\cdot A\]
This inequality explicitly quantifies the benefit of interconnection: even if the individual block mass $G_i$ is insufficient to resist $F_d$, the \textbf{collective mass}, modulated by $\eta $, ensures stability.

\subsection{Physical Interpretation of ${\eta }$}

The solidarization coefficient $\eta $ is critical to the applicability of USC. Its value reflects:

\begin{itemize}
\item  \textbf{Geometric interlock efficiency} (e.g., tetrapods vs. flat units),

\item  \textbf{Material cohesion} (rope strength, mesh resistance),

\item  \textbf{Assembly design} (anchorage, pre-tension, overlap zones),

\item  \textbf{Hydraulic conditions} (duration and frequency of jet loads).
\end{itemize}

Preliminary simulations and physical modeling suggest values ranging from 0.2 (weak linkage) to 0.8 (tight mesh configuration). Calibration of $\eta $ is thus essential for design applications.

\subsection{Design Implications}

Using USC enables designers to reduce the self-weight $G_i$ of each unit by up to 50\%, provided that the solidarization network is robust and properly quantified. This reduction leads to:

\begin{itemize}
\item  Lower material and transportation costs,

\item  Easier installation procedures,

\item  Enhanced adaptability in shallow or soft-soil environments.
\end{itemize}

The USC thereby shifts the design philosophy from \textbf{individual resistance} to \textbf{networked resilience}, consistent with modular engineering and sustainable construction trends.


\section{Comparative Analysis and Theoretical Application}

\subsection{Traditional Approach: Independent Unit Design}

Conventional scour protection designs treat each protective unit, whether it be a stone block, fascine mattress, or precast concrete element, as an \textbf{independent entity} resisting erosion via its own weight and friction. The design objective is to ensure that the drag force $F_d$, induced by the local flow (e.g., from ship propellers), does not exceed the unit's resisting weight $G_i$. This yields conservative requirements, often leading to oversized units (400--600 kg/block) with limited adaptability.
\[F_d<G_i\]
In environments with high turbulence and velocity gradients, such as close-range berthing of vessels over 10,000 TDW, this requirement may necessitate excessive use of material, increased installation time, and a greater carbon footprint.

\subsection{USC-Based Approach: Networked Resistance}

Under the USC model, resistance is not solely a function of a unit's own weight, but of its \textbf{collaborative mass} with neighboring units via structural interconnections (e.g., rope-mesh embedded geotextile):

\[F_d<G_i+\eta \cdot \sum{G_n}=G_{eff}\]
The solidarization coefficient $\eta $, even at moderate values (e.g., 0.5), effectively doubles the resisting capacity per unit if four adjacent neighbors are involved, allowing for a \textbf{reduction in individual block mass} without compromising safety.


\subsection{Theoretical Application -- Sample Calculation}

Assume the following design scenario:

\begin{itemize}
\item  Water density $\rho =1000\frac{kg}{m^3}$

\item  Drag coefficient $C_d=1.2$

\item  Projected area per block $A=0.15m^2$

\item  Flow velocity $v=3.5m/s$

\item  Number of connected neighbors = 4

\item  Neighbor weight $G_n=2500N$

\item  Solidarization coefficient $\eta =0.5$
\end{itemize}

\textbf{Drag force:}
\[F_d=\frac{1}{2}\cdot 1.2\cdot 1000\cdot {3.5}^2\cdot 0.15=1101.56N\]
\textbf{Effective resistance required:}
\[G_{eff}>1101.56N\]
If the block's own weight $G_i=1000N$, then the solidarized contribution is:
\[G_{eff}=1000+0.5\cdot (4\cdot 2500)=1000+5000=6000N\]
Thus, the USC-based design provides over \textbf{5 times} the minimum required resistance, allowing the block's own weight to be \textbf{reduced by more than 50\%} compared to conventional methods.

\subsection{Environmental and Practical Advantages}

\begin{itemize}
\item \textbf{Material efficiency}: Fewer raw materials required per block.

\item  \textbf{Installation benefits}: Easier handling, faster deployment.

\item  \textbf{Hydraulic performance}: Better energy dissipation due to flexible network response.

\item  \textbf{Carbon footprint}: Lower emissions from production and transport.
\end{itemize}

\subsection{Limitations and Calibration}

\begin{itemize}
\item Requires robust \textbf{testing or CFD modeling} to determine reliable $\eta $ values.

\item  Not suitable where \textbf{individual units may become detached} (e.g., high wave action without anchorage).

\item  Performance depends on the \textbf{maintenance of interconnectivity} over time.
\end{itemize}


\section{Discussion}

\subsection{Engineering Implications of the USC}

The \textbf{Universal Solidarization Criterion (USC)} introduces a design shift from independent-element resistance to \textbf{networked resilience}. This allows for:

\begin{itemize}
\item  \textbf{Optimized sizing} of individual units without compromising performance;

\item  \textbf{Cost-effective installations}, especially in remote or constrained locations;

\item  Improved safety margins through \textbf{distributed load sharing}, even under partial failure.
\end{itemize}

By quantifying the benefits of inter-unit connectivity via a scalar coefficient $\eta $, the USC transforms intuitive field practices (e.g., tying mattresses together) into a \textbf{formally applicable design framework}. This bridges the gap between empirical solutions and verifiable, code-compliant design logic.

\subsection{Interdisciplinary Relevance}

Though developed for scour protection in coastal and hydraulic engineering, the USC model holds conceptual value in other fields:

\begin{itemize}
\item  \textbf{Geotechnical engineering}: in the behavior of mechanically stabilized earth systems;

\item  \textbf{Structural engineering}: in seismic isolation layers or modular retaining structures;

\item  \textbf{Materials science}: in the design of composite systems with interlocked microstructures;

\item  \textbf{Urban resilience and disaster management}: where modular elements must collectively absorb impact or displacement.
\end{itemize}

The central insight, that \textbf{solidarity enhances resistance}, is a \textbf{universally transferable design principle}.

\subsection{Philosophical Dimensions: From Strength to Cohesion}

Beyond mathematics and material science, the USC evokes a \textbf{philosophical analogy}: resilience often emerges not from brute strength, but from \textbf{cooperation and integration}. This holds true in both physical systems and social constructs.

In engineered systems, distributed interdependence:

\begin{itemize}
\item  Adds \textbf{redundancy},

\item  Creates \textbf{stability under changing conditions},

\item  And encourages \textbf{adaptive behavior, }features often found in nature (e.g., honeycomb structures, fish schools, root networks).
\end{itemize}


The term ``solidarization'' intentionally captures this dual meaning, both \textbf{mechanical} and \textbf{sociological}.

\subsection{Need for Experimental Calibration}

While theoretical foundations of the USC are strong, future \textbf{physical model testing and CFD simulations} are required to:

\begin{itemize}
\item  Validate the assumed ranges of {\textbackslash}eta under varying configurations;

\item  Examine the role of \textbf{material type}, \textbf{connection stiffness}, and \textbf{hydrodynamic angle of attack};

\item  Establish \textbf{design charts or coefficients} for practical use in ports and waterways.
\end{itemize}


\section{Conclusions and Future Work}

This paper introduced the \textbf{Universal Solidarization Criterion (USC)} as a novel engineering framework to quantify and leverage the collective resistance of modular scour protection systems. Building on the limitations of classical sediment transport models---such as the Shields parameter and empirical bedload equations---the USC formalizes the beneficial effects of structural interconnectivity using the solidarization coefficient {\textbackslash}eta, thus redefining the threshold of dislodgement under hydraulic load.


The key contributions of this study include:

\begin{itemize}
\item  A \textbf{mathematical derivation} of a modular resistance model based on individual unit mass and solidarized neighbor contributions;

\item  A \textbf{comparative analysis} demonstrating that USC-based designs may enable a reduction in individual block mass by up to 50\% while maintaining performance;

\item  A conceptual framework with \textbf{cross-disciplinary applicability}, bridging hydraulics, structural resilience, and systems engineering;

\item  A \textbf{philosophical and functional shift} from individual resistance to distributed strength.
\end{itemize}

Future research should focus on:

\begin{itemize}
\item  \textbf{CFD simulations} using Reynolds-Averaged Navier--Stokes (RANS) solvers with calibrated propeller jet profiles to validate USC-based predictions;

\item  \textbf{Laboratory-scale physical models} to assess the mechanical behavior of solidarized systems under real flow conditions;

\item  \textbf{Development of practical design guidelines} for the application of USC in port infrastructure projects;

\item  Extension of the criterion to \textbf{3D granular matrices} and \textbf{anisotropic materials} with variable interconnection patterns.
\end{itemize}

Ultimately, the USC lays the foundation for an \textbf{evolutionary step in coastal engineering}, where modular systems act not in isolation but in solidarity, ensuring resilience, adaptability, and sustainable protection.

\nocite{*}

%
% Here's the list of references:
%
% \label{section:references}
\bibliography{ascexmpl-new}
%

\end{document}
